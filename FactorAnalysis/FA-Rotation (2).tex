%http://support.sas.com/publishing/pubcat/chaps/55129.pdf
%-------------------------------------------------------------------------------------------------------%
\subsection{Introduction to Rotation}


Factor patterns and factor loadings. 

After extracting the initial components, computer softwars
will create an unrotated factor pattern matrix. The rows of this matrix represent the variables
being analyzed, and the columns represent the retained components (these components would commonly be 
referred to as FACTOR1, FACTOR2 and so forth in the output).

The entries in the matrix are \textbf{\emph{factor loadings}}. A factor loading is a general term for a coefficient
that appears in a factor pattern matrix or a factor structure matrix. In an analysis that results in
oblique (correlated) components, the definition for a factor loading is different depending on
whether it is in a factor pattern matrix or in a factor structure matrix. 

However, the situation is simpler in an analysis that results in orthogonal components (as in the present case): In an
orthogonal analysis, factor loadings are equivalent to conventional bivariate correlations between the observed
variables and the components.
%-------------------------------------------------------------------------------------------------------%

\subsection{What is a Rotation}

Ideally, you would like to review the correlations between the variables and the
components and use this information to interpret the components; that is, to determine what
construct seems to be measured by component 1, what construct seems to be measured by
component 2, and so forth. Unfortunately, when more than one component has been retained in
an analysis, the interpretation of an unrotated factor pattern is usually quite difficult. To make
interpretation easier, you will normally perform an operation called a rotation. A rotation is a
linear transformation that is performed on the factor solution for the purpose of making the
solution easier to interpret.
%-------------------------------------------------------------------------------------------------------%

\subsection{Varimax Rotation}
A varimax rotation is an orthogonal rotation, meaning that
it results in uncorrelated components. Compared to some other types of rotations, a varimax
rotation tends to maximize the variance of a column of the factor pattern matrix (as opposed to a
row of the matrix). This rotation is probably the most commonly used orthogonal rotation in the
social sciences.

\subsection{Interpreting the Rotated Solution}

Interpreting a rotated solution means determining just what is measured by each of the retained
components. Briefly, this involves identifying the variables that demonstrate high loadings for a
given component, and determining what these variables have in common. Usually, a brief name
is assigned to each retained component that describes its content.
The first decision to be made at this stage is to decide how large a factor loading must be to be
considered ``large." 
Guidelines are providedd in statistical literature for testing the statistical significance of factor loadings. Given that this
is an introductory treatment of principal component analysis, however, simply consider a loading
to be “large” if its absolute value exceeds 0.40.

\subsection{SCRATCHOUT} 

%http://support.sas.com/publishing/pubcat/chaps/55129.pdf

\subsection{Creating Factor Scores or Factor-Based Scores}

Once the analysis is complete, it is often desirable to assign scores to each subject to indicate
where that subject stands on the retained components. For example, the two components
retained in the present study were interpreted as a financial giving component and an
acquaintance helping component. You may want to now assign one score to each subject to
indicate that subject’s standing on the financial giving component, and a different score to
indicate that subject’s standing on the acquaintance helping component. With this done, these
component scores could be used either as predictor variables or as criterion variables in
subsequent analyses.
Before discussing the options for assigning these scores, it is important to first draw a distinction
between factor scores versus factor-based scores. In principal component analysis, a factor
score (or component score) is a linear composite of the optimally-weighted observed variables.

Computer software can compute each subject’s factor scores for the two components
by

\begin{itemize}
\item determining the optimal regression weights
\item multiplying subject responses to the questionnaire items by these weights
\item summing the products.
\end{itemize}

The resulting sum will be a given subject’s score on the component of interest. Remember that a
separate equation, with different weights, is developed for each retained component.

A factor-based score, on the other hand, is merely a linear composite of the variables that
demonstrated meaningful loadings for the component in question. For example, in the preceding
analysis, items 4, 5, and 6 demonstrated meaningful loadings for the financial giving component.

Therefore, you could calculate the factor-based score on this component for a given subject by
simply adding together his or her responses to items 4, 5, and 6. Notice that, with a factor-based
score, the observed variables are not multiplied by optimal weights before they are summed.

