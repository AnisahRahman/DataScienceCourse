\section{Wald statistic}
Alternatively, when assessing the contribution of individual predictors in a given model, one may examine the significance of the Wald statistic. The Wald statistic, analogous to the t-test in linear regression, is used to assess the significance of coefficients. The Wald statistic is the ratio of the square of the regression coefficient to the square of the standard error of the coefficient and is asymptotically distributed as a chi-square distribution.[9]

\[W_j = \frac{B^2_j} {SE^2_{B_j}}\]

Although several statistical packages (e.g., SPSS, SAS) report the Wald statistic to assess the contribution of individual predictors, the Wald statistic has limitations. When the regression coefficient is large, the standard error of the regression coefficient also tends to be large increasing the probability of Type-II error. 

The Wald statistic also tends to be biased when data are sparse.

%--------------------------------------------------------%
