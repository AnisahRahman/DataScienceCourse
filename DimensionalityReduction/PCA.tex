\subsection{Principal Component Analysis}
Other types of procedures, such as principal component analysis and factor analysis, will transform the data.
Principal component analysis (PCA) is a mathematical procedure that uses an \textbf{orthogonal transformation} to convert a set of observations of possibly correlated independent variables into a set of values of artificial linearly uncorrelated variables called \textbf{principal components} (also called Factors in factor analysis). These principal components are used to represent the latent variables.

%-----------------------------------------%
\subsection{Orthogonal Transformation}
Orthogonal transformation is a Linear Algebra technique. It is not required to know it. 

Orthogonality can be crudely described as a synomym for perpendicularity. 

If two variables are orthogonal, they are essential uncorrelated.

The number of principal components is less than or equal to the number of original variables. 

This transformation is defined in such a way that the first principal component has the largest 
possible variance (that is, accounts for as much of the variability in the data as possible), and each 
succeeding component in turn has the highest variance possible under the constraint that it be orthogonal to (i.e., uncorrelated with) the preceding components.

%-----------------------------------------%

