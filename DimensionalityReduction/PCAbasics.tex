\section{Introduction to PCA}
Principal component analysis is a multivariate technique for examining relationships
among several quantitative variables. The choice between using factor analysis and
principal component analysis depends in part upon your research objectives. 

You should use the PRINCOMP procedure if you are interested in summarizing data and
detecting linear relationships. Plots of principal components are especially valuable
tools in exploratory data analysis. 

%----------------------------------------------------------------------------------%


Principal component analysis was originated by Pearson (1901) and later developed
by Hotelling (1933). The application of principal components is discussed by Rao
(1964), Cooley and Lohnes (1971), and Gnanadesikan (1977). Excellent statistical
treatments of principal components are found in Kshirsagar (1972), Morrison (1976),
and Mardia, Kent, and Bibby (1979).

%----------------------------------------------------------------------------------%


