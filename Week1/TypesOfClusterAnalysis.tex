%------------------------------------------------------------------------------%
Cluster analysis is a tool of discovery revealing associations and structure in data which, though not previously
evident, are sensible and useful when discovered. Importantly, CA enables new
cases to be assigned to classes for identification and diagnostic purposes; or find \textbf{\textit{exemplars}}
to represent classes.
%------------------------------------------------------------------------------%
\subsection{Types of Cluster Analysis}
There are three main types of cluster analysis.
\begin{itemize}
\item Hierarchical Clustering Analysis
\item Non-hierarchical Clustering Analysis (K-means clustering)
\item Two Step Clustering Analysis
\end{itemize}
%------------------------------------------------------------------------------%
Within hierarchical clustering analysis there are two subcategories: 
\begin{itemize}
\item Agglomerative (start from n clusters,to get to 1 cluster)
\item Divisive (start from 1 cluster, to get to $n$ cluster)
\end{itemize}
%------------------------------------------------------------------------------%
