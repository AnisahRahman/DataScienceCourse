%---------------------------%
\section{RapidMiner}

The Community Edition of RapidMiner (formerly "Yale") is an open source toolkit for data mining. Its strengths reside in part in its ability to easily define analytical steps (especially when compared with R), and in generating graphs more easily than e.g., R, or more effectively than MS Excel. 


What is it for?

RapidMiner is well suited for analyzing data generated by high-throughput instruments, e.g., genotyping, proteomics, and mass spectrometry. 


%---------------------------%
\subsection{Example applications:}

\begin{itemize}
\item Bypassing its data mining functions and simply having RapidMiner generate figures.
\item Exploring your data in "super-Excel" fashion ("knowledge discovery").
\item Constructing custom data analysis workflows.
\item Calling RapidMiner functions from programs written in other languages/systems (e.g., Perl).
\end{itemize}

Notable selected features of RapidMiner:

\begin{itemize}
\item Broad collection of data mining algorithms such as decision trees and self-organization maps.
\item Sophisticated graphics, such as overlapping histograms, tree charts and 3D scatter plots.
\item Many varied plugins, such as a text plugin for doing text analysis.
\end{itemize}
