\documentclass[12pt]{article}


\usepackage{framed}

\begin{document}
This is an introductory course in machine learning (ML) that covers the basic theory, 
algorithms, and applications. ML is a key technology in Big Data, and in many financial,
 medical, commercial, and scientific applications. It enables computational systems to 
automatically learn how to perform a desired task based on information extracted from the data. 

ML has become one of the hottest fields of study today, taken up by undergraduate and 
graduate students from 15 different majors at Caltech. This course balances theory and 
practice, and covers the mathematical as well as the heuristic aspects. 

\begin{description}
\item[Lecture 1] The Learning Problem
\item[Lecture 2] Is Learning Feasible?
\item[Lecture 3] The Linear Model I
\item[Lecture 4] Error and Noise
\item[Lecture 5] Training versus Testing
\item[Lecture 6] Theory of Generalization
\item[Lecture 7] The VC Dimension
\item[Lecture 8] Bias-Variance Tradeoff
\item[Lecture 9] The Linear Model II
\item[Lecture 10] Neural Networks
\item[Lecture 11] Overfitting
\item[Lecture 12] Regularization
\item[Lecture 13] Validation
\item[Lecture 14] Support Vector Machines
\item[Lecture 15] Kernel Methods
\item[Lecture 16] Radial Basis Functions
\item[Lecture 17] Three Learning Principles
\item[Lecture 18] Epilogue
\end{description}


\end{document}
