
Computing


Computing

Object oriented programming

Management information systems (MIS)


Object oriented programming

Object oriented programming, or OOP, is one model of programming language. Unlike other examples of computer programming language, object oriented programming focuses on the use of objects instead of actions in order to carry out tasks. This use of objects to design applications instead of actions also involves taking an approach that is more mindful of data and less concerned with logic, which is more commonly the case in other paradigms.


This different approach taken by object oriented programming means that the view of objects and actions is reversed. The emphasis is on the objects themselves rather than on the execution of tasks that employ the objects. In like manner, the structure of object oriented programming does not consider deciding on how to employ the logic, but on the definition of the data that will be used in the programming.


Designing computer programs with the approach of object oriented programming begins with defining the objects that are to be manipulated by the program. Once the objects are identified, the programmer will begin to identify the relationship between each object. This process is usually referred to as data modeling. Essentially, the programmer is seeking to place the objects into a classification, thus helping to define the data that is part of the inheritance brought to the task by each object. In fact, the process of defining these classes and subclasses of data is normally called inheritance.


Object oriented programming also helps to sort objects in a manner that allows for the phenomenon of polymorphism to take place. That is, different objects will be able to respond to a common message, but each in a different way that is unique to that object. At the same time, object oriented programming allows for the encapsulation of an object, effectively hiding or protecting the data associated with the object from easy view without security access.


One of the advantages of object oriented programming is that the process makes good use of modularity. That is, objects and tasks are grouped in a way that each module is capable of independent consideration. This can be a great help when making enhancements to a program, as modularity makes it possible to address the task of making alternations to the setup of one portion of the programming without impacting the structure and function of the other modules.

Management information systems (MIS) 

Management information systems (MIS) are a combination of hardware and software used to process information automatically. Commonly, MIS are used within organizations to allow many individuals to access and modify information. In most situations, the management information system mainly operates behind the scenes, and the user community is rarely involved or even aware of the processes that are handled by the system.


A computer system used to process orders for a business could be considered a management information system because it is assisting users in automating processes related to orders. Other examples of modern management information systems are websites that process transactions for an organization or even those that serve support requests to users. A simple example of a management information system might be the support website for a product, because it automatically returns information to the end user after some initial input is provided.


Online bill pay at a bank also qualifies as a management information system — when a bill is scheduled to be paid, the user has provided information for the system to act against. The management information system then processes the payment when the due date approaches. The automated action taken by the online system is to pay the bill as requested. Since the bills within an online bill pay system can be scheduled to be automatically paid month after month, the user is not required to provide further information. Many times, the bill pay system will also produce an email for the user to let him know that the action has occurred and what the outcome of the action was.


Management information systems typically have their own staff whose function it is to maintain existing systems and implement new technologies within a company. These positions are often highly specialized, allowing a team of people to focus on different areas within the computer system. In recent years, colleges and universities have begun offering entire programs devoted to management information systems. In these programs, students learn how to manage large interconnected computer systems and troubleshoot the automation of these management information systems.


Many people use management information systems every day without thinking about the actual system they are using. The individual will see a website and enter information with the expectation that a specific action will happen; these websites, just like the accounting systems used by large corporations, act as management information systems to automate the process.


