%---------------------------%
\section{KDD : Knowledge Discovery in Databases}

KDD refers to the overall process of gleaning knowledge from data.


Knowledge discovery in databases (KDD), also referred to as (statistical) data mining, is an area of common interest to researchers in machine discovery, statistics, databases, knowledge acquisition, machine learning, data visualization, high performance computing, and knowledge-based systems.


Data Mining term is used interchangeably with KDD. In reality, it is one of the steps in the whole process of knowledge discovery in databases.


Data Mining needs a well defined business case and a diligent data preparation and has to be followed with a detailed evaluation of the discovery results.


KDD refers to the overall process of discovering useful knowledge from data, and data mining refers to a particular step in this process. Data mining is the application of specific algorithms for extracting patterns from data.


At an abstract level, the KDD field is concerned with the development of methods and techniques for making sense of data. The basic problem addressed by the KDD process is one of mapping low-level data (which are typically too voluminous to understand and digest easily) into other forms that might be more compact (for example, a short report), more abstract (for example, a descriptive approximation or model of the process that generated the data), or more useful (for example, a predictive model for estimating the value of future cases).


KDD is an attempt to address a problem that the digital information era made a fact of life for all of us: data overload.

\end{document}
%--------------------------------------%
