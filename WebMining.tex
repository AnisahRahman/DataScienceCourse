\begin{document}
\maketitle

\section{Web-mining}




More than ever, entities and individuals alike are using the World Wide Web to conduct a host of business and personal transactions. As a result, companies are increasingly employing Web data mining tools and techniques in order to find ways to improve their bottom lines and grow their customer base. Web data mining involves the process of collecting and summarizing data from a Web site’s hyperlink structure, page content, or usage log in order to identify patterns. Using Web data mining, a company can identify a potential competitor, improve customer service, or target customer needs and expectations. A government agency may also seek to uncover terrorist threats or other criminal activities through the use of a Web data mining application.

 

Some common Web data mining techniques include Web content mining, Web usage mining, and Web structure mining. Web content mining examines the subject matter of a Web site. For example, Web content miners may analyze a site's audio, text, images, and video features. Web content miners typically focus on a site’s textual information more than other site features. Natural language processing and information retrieval are two data mining techniques often used by Web content miners.

 

Web usage mining is usually an automated process whereby Web servers collect and report user access patterns in server access logs. A company may, for example, use a Web usage data mining tool to report on server access logs and user registration information in order to create a more effective Web site structure. Web structure mining studies the node and connection structure of Web sites. It can be useful in identifying similarities and relationships that exist among different Web sites. Web structure mining often involves uncovering patterns from hyperlinks or pulling out document structures on a Web page.

 

Two general data mining techniques that can be employed by Web data miners are data mining association analysis and data mining regression. Data mining association analysis helps uncover noteworthy relationships buried in large data sets. Data mining regression is a statistical technique whereby mathematical formulas are used to predict future results, such as profit margins, house values, or sales figures.

Data mining software vendors offer Web data mining tools that can pull out predictive information from large quantities of data. Businesses often use these software mining tools to analyze specific data sets regarding consumer behavior. Using the results of the data analysis, companies are able to forecast future business trends.

 

Data mining (DMM), also called \textbf{\textit{Knowledge-Discovery in Databases (KDD) }}or Knowledge-Discovery and Data Mining, is the process of automatically searching large volumes of data for patterns using tools such as classification, association rule mining, clustering, etc. Data mining is a complex topic and has links with multiple core fields such as computer science and adds value to rich seminal computational techniques from statistics, information retrieval, machine learning and pattern recognition. 


Data mining has been defined as "the nontrivial extraction of implicit, previously unknown, and potentially useful information from data"  and "the science of extracting useful information from large data sets or databases" . It involves sorting through large amounts of data and picking out relevant information. It is usually used by businesses, intelligence organizations, and financial analysts, but is increasingly used in the sciences to extract information from the enormous data sets generated by modern experimental and observational methods. Metadata, or data about a given data set, are often expressed in a condensed data mine-able format, or one that facilitates the practice of data mining. Common examples include executive summaries and scientific abstracts. 


Although data mining is a relatively new term, the technology is not. Companies for a long time have used powerful computers to sift through volumes of data such as supermarket scanner data, and produce market research reports. Continuous innovations in computer processing power, disk storage, and statistical software are dramatically increasing the accuracy and usefulness of analysis. Data mining identifies trends within data that go beyond simple analysis. Through the use of sophisticated algorithms, users have the ability to identify key attributes of business processes and target opportunities. The term data mining is often used to apply to the two separate processes of knowledge discovery and prediction. Knowledge discovery provides explicit information that has a readable form and can be understood by a user. Forecasting, or predictive modeling provides predictions of future events and may be transparent and readable in some approaches (e.g. rule based systems) and opaque in others such as neural networks. Moreover, some data mining systems such as neural networks are inherently geared towards prediction and pattern recognition, rather than knowledge discovery. 


The term "data mining" is often used incorrectly to apply to a variety of other processes besides data mining. In many cases, applications may claim to perform "data mining" by automating the creation of charts or graphs with historic trends and analysis. Although this information may be useful and timesaving, it does not fit the traditional definition of data mining, as the application performs no analysis itself and has no understanding of the underlying data. Instead, it relies on templates or pre-defined macros (created either by programmers or users) to identify trends, patterns and differences. A key defining factor for true data mining is that the application itself is performing some real analysis. In almost all cases, this analysis is guided by some degree of user interaction, but it must provide the user some insights that are not readily apparent through simple slicing and dicing. Applications that are not to some degree self-guiding are performing data analysis, not data mining.
