\documentclass[a4paper,12pt]{article}
%%%%%%%%%%%%%%%%%%%%%%%%%%%%%%%%%%%%%%%%%%%%%%%%%%%%%%%%%%%%%%%%%%%%%%%%%%%%%%%%%%%%%%%%%%%%%%%%%%%%%%%%%%%%%%%%%%%%%%%%%%%%%%%%%%%%%%%%%%%%%%%%%%%%%%%%%%%%%%%%%%%%%%%%%%%%%%%%%%%%%%%%%%%%%%%%%%%%%%%%%%%%%%%%%%%%%%%%%%%%%%%%%%%%%%%%%%%%%%%%%%%%%%%%%%%%
\usepackage{eurosym}
\usepackage{vmargin}
\usepackage{amsmath}
\usepackage{graphics}
\usepackage{epsfig}
\usepackage{subfigure}
\usepackage{fancyhdr}
%\usepackage{listings}
\usepackage{framed}
\usepackage{graphicx}

\setcounter{MaxMatrixCols}{10}
%TCIDATA{OutputFilter=LATEX.DLL}
%TCIDATA{Version=5.00.0.2570}
%TCIDATA{<META NAME="SaveForMode" CONTENT="1">}
%TCIDATA{LastRevised=Wednesday, February 23, 2011 13:24:34}
%TCIDATA{<META NAME="GraphicsSave" CONTENT="32">}
%TCIDATA{Language=American English}

\pagestyle{fancy}
\setmarginsrb{20mm}{0mm}{20mm}{25mm}{12mm}{11mm}{0mm}{11mm}
\lhead{MA4128} \rhead{Mr. Kevin O'Brien}
\chead{Advanced Data Modelling}
%\input{tcilatex}


% http://www.norusis.com/pdf/SPC_v13.pdf
\begin{document}


%SESSION 1: Hierarchical Clustering
% Hierarchical clustering - dendrograms
% Divisive vs. agglomerative methods
% Different linkage methods

%SESSION 2: K-means Clustering

\tableofcontents
\newpage

\section{Introduction to Cluster Analysis}


%Introduction
Cluster analysis is a major technique for classifying a large volumes of information into
manageable meaningful piles. Cluster analysis is a data reduction tool that creates subgroups that are
more manageable than individual data items. In other words cluster analysis is an exploratory data analysis tool which aims at sorting different objects into groups in a way that the degree of association between two objects is maximal if they belong to the same group and minimal otherwise. Given the above, cluster analysis can be used to discover structures in data without providing an explanation/interpretation. In other words, cluster analysis simply discovers structures in data without explaining why they exist. Like factor analysis, it examines the full complement
of inter-relationships between variables. Both cluster analysis (and later, discriminant
analysis) are concerned with classification.

However, the latter requires prior knowledge of membership of each cluster in order to classify new cases. In cluster analysis
there is no prior knowledge about which elements belong to which clusters. The grouping
or clusters are defined through an analysis of the data. Subsequent multivariate analyses
can be performed on the clusters as groups.

\textbf{\textit{A cluster is a group of relatively homogeneous cases or observations.}}


The term cluster analysis encompasses a number of different algorithms and methods for grouping objects of similar kind into respective categories.A general question facing researchers in many areas of inquiry is how to organize observed data into meaningful structures, that is, to develop \textbf{\emph{taxonomies}}.

\subsection{Applications of Cluster Analysis}

We deal with clustering in almost every aspect of daily life. For example, a group of diners sharing the same table in a restaurant may be regarded as a cluster of people. In food stores items of similar nature, such as different types of meat or vegetables are displayed in the same or nearby locations. There is a countless number of examples in which clustering plays an important role. Clustering techniques have been applied to a wide variety of scientific research problems. For example, in the field of medicine, clustering diseases, cures for diseases, or symptoms of diseases can lead to very useful taxonomies. In the field of psychiatry, the correct diagnosis of clusters of symptoms such as paranoia, schizophrenia, etc. is essential for successful therapy. In archeology, researchers have attempted to establish taxonomies of stone tools, funeral objects, etc. by applying cluster analytic techniques. According to the modern system employed in biology, man belongs to the primates, the mammals, the amniotes, the vertebrates, and the animals.

Note how in this classification, the higher the level of aggregation the less similar are the members in the respective class. Man has more in common with all other primates (e.g., apes) than it does with the more "distant" members of the mammals (e.g., dogs), etc.

In general, whenever we need to classify a "mountain" of information into manageable meaningful piles, cluster analysis is of great utility.

\end{document}
