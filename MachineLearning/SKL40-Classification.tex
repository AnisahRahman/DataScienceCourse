
\section{Classifying Data with scikit-learn}
This chapter will cover the following topics:
\begin{enumerate}
\item  Doing basic classifications with Decision Trees
\item  Tuning a Decision Tree model
\item  Using many Decisions Trees – random forests
\item  Tuning a random forest model
\item  Classifying data with support vector machines
\item  Generalizing with multiclass classification
\item  Using LDA for classification
\item  Working with QDA – a nonlinear LDA
\item  Using Stochastic Gradient Descent for classification
\item  Classifying documents with Naïve Bayes
\item  Label propagation with semi-supervised learning
\end{enumerate}
\subsection{Introduction}
Classification can be very important in a lot of contexts. For example, if we want to automate
some decision-making process, we can utilize classification. In cases where we need to
investigate a fraud, there are so many transactions that it is impractical for a person to
check all of them. Therefore, we can automate such decisions with classification.
\end{document}
%===================================================%
%% Classifying Data with scikit-learn
%% 120
