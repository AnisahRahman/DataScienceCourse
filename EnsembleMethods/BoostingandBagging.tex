% http://stats.stackexchange.com/questions/18891/bagging-boosting-and-stacking-in-machine-learning
% http://people.cs.pitt.edu/~milos/courses/cs2750-Spring04/lectures/class23.pdf
% http://www.jstatsoft.org/v54/i02/paper


Bagging (stands for Bootstrap Aggregation) is the way decrease the variance of your prediction by generating additional data for training from your original dataset using combinations with repetitions to produce multisets of the same cardinality/size as your original data. By increasing the size of your training set you can't improve the model predictive force, but just decrease the variance, narrowly tuning the prediction to expected outcome.

Boosting is a an approach to calculate the output using several different models and then average the result using a weighted average approach. By combining the advantages and pitfalls of these approaches by varying your weighting formula you can come up with a good predictive force for a wider range of input data, using different narrowly tuned models.

Stacking is a similar to boosting: you also apply several models to you original data. The difference here is, however, that you don't have just an empirical formula for your weight function, rather you introduce a meta-level and use another model/approach to estimate the input together with outputs of every model to estimate the weights or, in other words, to determine what models perform well and what badly given these input data.

As you see, these all are different approaches to combine several models into a better one, and there is no single winner here: everything depends upon your domain and what you're going to do. You can still treat stacking as a sort of more advances boosting, however, the difficulty of finding a good approach for your meta-level makes it difficult to apply this approach in practice.

Short examples of each:

Bagging: Ozone data.
Boosting: is used to improve optical character recognition (OCR) accuracy.
Stacking: is used in K-fold cross validation algorithms.
